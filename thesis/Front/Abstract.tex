% !TEX root = ../Thesis.tex
\chapter{Abstract}

Symbolic execution is a powerful program analysis technique widely used for vulnerability discovery and test case generation. However, its practical application is often hampered by scalability issues, primarily due to the "path explosion problem" where the number of possible execution paths grows exponentially with program complexity. This thesis addresses this fundamental challenge by proposing an optimized approach to symbolic execution that integrates taint analysis and path prioritization.

The core idea is to move away from uniform exploration of all program paths towards a more targeted analysis, focusing on paths that are most relevant to security-critical aspects. Specifically, this work prioritizes paths originating from memory allocations and user inputs, as these are common sources of vulnerabilities. By leveraging taint analysis, the system identifies and tracks data originating from these critical sources, allowing the symbolic execution engine to concentrate its efforts on paths influenced by such "tainted" data, while ignoring paths with no dependency on external inputs.

The effectiveness of this optimization is evaluated through a comparative analysis, examining criteria such as runtime efficiency, the number and relevance of discovered program paths, and the identification of potential vulnerabilities. This approach aims to significantly improve the efficiency and applicability of symbolic execution for large and complex software systems.
