% !TEX root = ../Thesis.tex
\chapter{Conclusion}

This thesis presented TraceGuard, a novel approach that integrates dynamic taint analysis with symbolic execution to address the fundamental path explosion problem in vulnerability discovery. By prioritizing execution paths based on their interaction with user-controlled data, TraceGuard transforms symbolic execution from uniform exploration into intelligent, security-focused analysis.

The work delivered a comprehensive taint-guided symbolic execution system. We established a theoretical framework integrating taint analysis with symbolic execution through real-time path prioritization, implementing a three-tier scoring system that guides exploration toward vulnerability-prone paths. TraceGuard was implemented using the Angr framework, featuring function hooking, symbolic taint tracking, and adaptive state management with architectural support, primarily for AMD64 systems.

The experimental validation demonstrated TraceGuard's effectiveness across seven benchmark programs, achieving 100\% vulnerability detection while finding $5 \times$ more vulnerabilities than classical symbolic execution. TraceGuard explores only 36.8\% to 75.0\% of the basic blocks while maintaining significantly improved vulnerability detection, demonstrating the advantage of security-focused exploration.

This work addresses a fundamental limitation in symbolic execution research—the lack of security-aware path prioritization. Previous approaches treated all execution paths equally, leading to inefficient resource allocation. The approach bridges dynamic and static analysis techniques, effectively guiding symbolic exploration toward security-relevant program regions.

The evaluation used synthetic test programs with predefined taint sources. TraceGuard's effectiveness depends on accurate taint source identification, currently limited to standard input functions. The approach was validated on AMD64 C/C++ binaries.

TraceGuard successfully demonstrates that integrating taint analysis with symbolic execution through intelligent path prioritization significantly improves vulnerability discovery while maintaining computational efficiency. The approach establishes a foundation for security-aware program analysis that prioritizes vulnerability discovery over general code coverage, providing a platform for continued research in symbolic execution optimization and practical security analysis.
