% !TEX root = ../Thesis.tex
\chapter{Related Work}

This chapter examines specific approaches to improving symbolic execution effectiveness for vulnerability discovery, focusing on methodological innovations that address different aspects of the scalability challenge.

\section{Compositional Analysis: MACKE}

MACKE \cite{ognawala_macke_2016} introduces a compositional approach that performs symbolic execution on individual program functions in isolation, then combines results using static analysis and inter-procedural path feasibility checking. The tool employs a three-step process: first analyzing functions separately to achieve higher coverage, then using compositional reasoning to reduce false positives, and finally assigning severity scores to reported vulnerabilities. MACKE demonstrated superior coverage compared to forward symbolic execution on real-world programs including Flex, Grep, and Bzip2, while providing significantly fewer false positives than pure static analysis approaches.

\section{Compilation-Based Execution: SymCC}

SymCC \cite{poeplau_symbolic_2020} revolutionized symbolic execution performance through a compilation-based approach that embeds symbolic execution directly into program binaries during compilation. Rather than interpreting intermediate representations at runtime, SymCC inserts symbolic computation code alongside concrete execution, achieving significant performance improvements with 10-12 $\times$ speedup over traditional approaches like KLEE and QSYM. This optimization enabled higher coverage analysis and led to the discovery of vulnerabilities in heavily tested projects such as OpenJPEG.
