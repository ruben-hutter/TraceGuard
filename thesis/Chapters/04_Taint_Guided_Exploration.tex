% !TEX root = ../Thesis.tex
\chapter{Taint-Guided Exploration}\label{ch:taint_guided_exploration}

Having established the theoretical foundations in Chapter~\ref{ch:background} and surveyed existing approaches in Chapter~\ref{ch:related_work}, this chapter presents the conceptual framework and algorithmic design of TraceGuard's taint-guided symbolic execution strategy. Rather than exploring all possible execution paths uniformly, TraceGuard prioritizes paths based on their interaction with potentially malicious user input, fundamentally addressing the path explosion problem through intelligent exploration guidance.

The core insight underlying this approach is that security vulnerabilities are significantly more likely to occur in code paths that process external, user-controlled data. By tracking taint flow from input sources and using this information to guide symbolic execution, TraceGuard focuses computational resources on security-relevant program regions while avoiding exhaustive exploration of paths that operate solely on trusted internal data.

\section{Core Approach}

TraceGuard operates as a specialized program built on the Angr framework that transforms symbolic execution from exhaustive path exploration into a security-focused analysis. The approach centers on four key mechanisms that work together to prioritize execution paths based on their interaction with potentially malicious user input.

\textbf{Hook-Based Taint Detection:} The system intercepts function calls during symbolic execution to identify when external data enters the program. Input functions like \texttt{fgets} and \texttt{scanf} are immediately flagged as taint sources, while other functions are monitored for tainted parameter usage.

\textbf{Symbolic Taint Tracking:} Tainted data is tracked through unique symbolic variable names and memory region mappings. When input functions create symbolic data, the variables receive distinctive ``\texttt{taint\_source\_}'' prefixes that persist throughout symbolic execution.

\textbf{Dynamic State Prioritization:} Each symbolic execution state receives a taint score based on its interaction with tainted data. States are classified into three priority levels that determine exploration order: high priority (score $\geq \tau_{high}$), medium priority ($\tau_{medium} \leq$ score $< \tau_{high}$), and normal priority (score $< \tau_{medium}$).

\textbf{Exploration Boundaries:} Multiple complementary techniques prevent path explosion: execution length limits, loop detection, and graduated depth penalties that naturally favor shorter paths to vulnerability-triggering conditions.

\vspace{1em}

Throughout the following algorithms, we use configurable parameters to maintain generality: $\alpha_{input}$ represents the score bonus for input function interactions, $\beta_{tainted}$ denotes the bonus for execution within tainted functions, $\gamma_{penalty}$ specifies the depth penalty multiplication factor, $\delta_{threshold}$ defines the depth threshold for penalty application, $\sigma_{min}$ sets the minimum exploration score, $\tau_{high}$ and $\tau_{medium}$ establish the priority classification thresholds, and $k$ determines the maximum number of active states. In our implementation, these parameters are set to $\alpha_{input} = 5.0$, $\beta_{tainted} = 3.0$, $\gamma_{penalty} = 0.95$, $\delta_{threshold} = 200$, $\sigma_{min} = 1.0$, $\tau_{high} = 6.0$, $\tau_{medium} = 2.0$, and $k = 15$.

\section{Taint Source Recognition}\label{sec:taint_source_recognition}

TraceGuard identifies taint sources by hooking functions during program analysis. This hook-based approach enables runtime detection of external data entry points without requiring complex static analysis.

\begin{algorithm}[H]
\caption{Function Hooking Strategy}\label{alg:function_hooking}
\begin{algorithmic}[1]
\Require Program binary $P$
\State $CFG \gets \textsc{BuildControlFlowGraph}(P)$
\State $InputFunctions \gets \{\text{fgets, scanf, read, gets}\}$
\ForAll{function $f$ in $CFG$}
    \If{$f.name \in InputFunctions$}
        \State $\textsc{InstallInputHook}(f)$
    \Else
        \State $\textsc{InstallGenericHook}(f)$
    \EndIf
\EndFor
\end{algorithmic}
\end{algorithm}

The system uses two types of hooks: input function hooks that immediately mark data as tainted, and generic hooks that check whether function parameters contain tainted data. This dual approach ensures both taint introduction and propagation are monitored throughout execution.

Input functions receive special treatment because they represent the primary vectors for external data entry. When these functions are called, the system automatically creates tainted symbolic data and registers the associated memory regions as containing potentially malicious content.

\section{Dynamic Taint Tracking}\label{sec:dynamic_taint_tracking}

TraceGuard tracks taint propagation through two complementary mechanisms: symbolic variable naming and memory region mapping. This approach ensures taint information persists across function calls and memory operations.

\begin{algorithm}[H]
\caption{Taint Introduction at Input Functions}
\begin{algorithmic}[1]
\Require Function call to input function $f$, State $s$
\State $data \gets \textsc{CreateSymbolicData}(\text{taint\_source\_} + f.name)$
\State $s.globals[\text{taint\_score}] \gets s.globals[\text{taint\_score}] + \alpha_{input}$
\State $s.globals[\text{tainted\_functions}].add(f.name)$
\If{$f$ involves memory allocation}
    \State $buffer\_addr \gets \textsc{GetBufferAddress}(s)$
    \State $buffer\_size \gets \textsc{GetBufferSize}(s)$
    \State $s.globals[\text{tainted\_regions}].add((buffer\_addr, buffer\_size))$
\EndIf
\State \Return $data$
\end{algorithmic}
\end{algorithm}

Symbolic variable naming creates a persistent taint identifier that follows data through symbolic operations. Memory region tracking maintains a mapping of tainted buffer addresses and sizes, enabling taint detection when pointers reference previously tainted memory locations.

\begin{algorithm}[H]
\caption{Taint Status Check}
\begin{algorithmic}[1]
\Require State $s$, Variable or address $target$
\If{$target$ is symbolic variable}
    \State \Return $\text{taint\_source\_} \in target.name$
\ElsIf{$target$ is memory address}
    \ForAll{$(addr, size)$ in $s.globals[\text{tainted\_regions}]$}
        \If{$addr \leq target < addr + size$}
            \State \Return $\textsc{True}$
        \EndIf
    \EndFor
\EndIf
\State \Return $\textsc{False}$
\end{algorithmic}
\end{algorithm}

Additionally, memory region tracking maintains a mapping of tainted buffer addresses and sizes, enabling taint detection when pointers reference previously tainted memory locations.

\section{Path Prioritization}\label{sec:path_prioritization}

TraceGuard implements a three-tier prioritization system that classifies symbolic execution states based on their calculated taint scores. This classification determines exploration order to focus computational resources on security-relevant paths.

\begin{algorithm}[H]
\caption{State Classification and Prioritization}
\begin{algorithmic}[1]
\Require Active states $\mathcal{S}$, Thresholds $\tau_{high}$, $\tau_{medium}$
\State $scored\_states \gets []$
\ForAll{state $s \in \mathcal{S}$}
    \State $score \gets \textsc{CalculateTaintScore}(s)$
    \State $scored\_states.append((score, s))$
\EndFor
\State $P_{high} \gets \{s : score \geq \tau_{high}\}$
\State $P_{medium} \gets \{s : \tau_{medium} \leq score < \tau_{high}\}$ 
\State $P_{normal} \gets \{s : score < \tau_{medium}\}$
\State exploration\_queue $\gets P_{high} + P_{medium} + P_{normal}$
\State \Return first $k$ states from exploration\_queue
\end{algorithmic}
\end{algorithm}

The score calculation combines multiple factors to assess security relevance. Base scores come from taint interactions tracked by function hooks, with additional bonuses for execution within previously identified tainted functions and penalty reductions for excessive execution depth.

\begin{algorithm}[H]
\caption{Taint Score Calculation}
\begin{algorithmic}[1]
\Require State $s$
\State $score \gets \max(s.globals[\text{taint\_score}], \sigma_{min})$
\If{current function $\in$ tainted functions}
    \State $score \gets score + \beta_{tainted}$
\EndIf
\If{execution depth $> \delta_{threshold}$}
    \State $score \gets score \times \gamma_{penalty}$
\EndIf
\State \Return $score$
\end{algorithmic}
\end{algorithm}

High-priority states typically represent paths directly processing user input or executing within security-critical functions. Medium-priority states show moderate taint relevance, while normal-priority states primarily handle untainted data. The system limits active states to prevent path explosion while maintaining adequate exploration coverage.

\subsection{Adaptive State Pool Management}

A critical component of TraceGuard's practical viability lies in its adaptive state pool management strategy, which prevents path explosion while maintaining exploration effectiveness. The system employs a bounded exploration approach that dynamically adjusts the active state pool based on both computational constraints and taint score distributions.

\textbf{Bounded Exploration Principle:} Rather than allowing unlimited state proliferation, TraceGuard maintains a fixed upper bound $k$ on concurrent active states. This constraint transforms the potentially infinite symbolic execution search space into a manageable, resource-bounded exploration process. The bound $k$ represents a balance between exploration thoroughness and computational tractability, typically set to a small constant based on empirical analysis of memory usage and solver performance.

\textbf{Dynamic State Replacement:} When the exploration encounters new states that would exceed the bound $k$, the system employs a replacement strategy based on taint scores. New states are only admitted to the active pool if their taint scores exceed those of current low-priority states. This ensures that computational resources remain focused on the most security-relevant execution paths, even as the program exploration discovers new branches.

\textbf{Priority-Based Pruning:} The state pruning mechanism operates according to the established three-tier priority system. When resource limits are reached, normal-priority states are pruned first, followed by medium-priority states if necessary. High-priority states are preserved except in extreme cases where all active states achieve high-priority classification, at which point fine-grained score comparisons determine pruning order.

This adaptive approach ensures that TraceGuard maintains bounded computational requirements while maximizing the security relevance of explored paths, addressing both the theoretical challenge of path explosion and the practical constraints of finite computational resources.

\section{Exploration Depth Control and Vulnerability Probability}\label{sec:exploration_depth_control}

TraceGuard prevents path explosion through multiple complementary techniques that limit exploration depth while maintaining sufficient coverage for vulnerability discovery. A fundamental principle underlying this approach is the inverse relationship between execution depth and vulnerability probability.

The preference for shorter paths in vulnerability discovery is grounded in both theoretical security principles and empirical evidence from vulnerability research~\cite{schwartz_all_2010}. Security vulnerabilities typically manifest near the boundary between external input and internal program logic, where insufficient validation or sanitization allows malicious data to corrupt program state. As execution depth increases beyond these initial input processing stages, several factors reduce vulnerability probability: (1) input data has undergone additional validation and transformation steps, (2) the program state becomes more complex and harder for attackers to predict and control, and (3) deeper code paths typically receive more thorough testing during development.

Research on real-world vulnerability databases demonstrates that critical security flaws such as buffer overflows and injection attacks are statistically more likely to occur in shallow call stacks near input sources than in deeply nested program logic. This observation aligns with attack surface theory, which suggests that the most accessible vulnerabilities are those that can be triggered with minimal program state setup, making them both more discoverable by automated tools and more attractive to attackers.

\begin{algorithm}[H]
\caption{Progressive Depth Penalties}
\begin{algorithmic}[1]
\Require State $s$ with execution depth $d$
\If{$d > \delta_{high}$}
    \State $s.score \gets s.score \times \gamma_{high}$
\ElsIf{$d > \delta_{medium}$}
    \State $s.score \gets s.score \times \gamma_{medium}$
\EndIf
\end{algorithmic}
\end{algorithm}

The depth penalty system gradually reduces state scores as execution depth increases, naturally prioritizing shorter paths that are more likely to trigger vulnerabilities quickly. This graduated approach avoids abrupt path termination while steering exploration toward more promising regions of the program space. The system employs configurable depth thresholds ($\delta_{high}$, $\delta_{medium}$) and penalty factors ($\gamma_{high}$, $\gamma_{medium}$) to balance thorough exploration with computational efficiency.

Beyond depth penalties, TraceGuard coordinates multiple exploration control mechanisms to manage path explosion effectively. These include execution length limitations to prevent infinite loops, cycle detection to avoid repetitive exploration patterns, and adaptive state management that maintains an optimal number of active states based on available computational resources.
