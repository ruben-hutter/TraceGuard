% !TEX root = ../Thesis.tex
\chapter{Background}\label{ch:background}

This chapter establishes the theoretical foundations necessary for understanding the taint-guided symbolic execution optimization presented in this thesis. We examine symbolic execution, program vulnerability analysis, taint analysis, control flow analysis, and the Angr\footnote{https://angr.io/} framework.

\section{Symbolic Execution}

Symbolic execution is a program analysis technique that explores execution paths by using symbolic variables instead of concrete inputs. The program state consists of symbolic variables, path constraints, and a program counter. When execution encounters a conditional branch, the engine explores both branches by adding appropriate constraints to the path condition.

A fundamental challenge in symbolic execution is the path explosion problem. As program complexity increases, the number of possible execution paths grows exponentially, making exhaustive exploration computationally intractable. This scalability issue particularly affects real-world applications with complex control flow structures and deep function call hierarchies. Research has shown that symbolic execution tools designed to optimize statement coverage often fail to cover potentially vulnerable code due to complex system interactions and scalability issues of constraint solvers~\cite{schwartz_all_2010}.

Traditional symbolic execution typically employs a forward approach, starting from the program's entry point and exploring paths toward potential targets. However, this method may struggle to reach deeply nested functions or specific program locations of interest. Backward symbolic execution, conversely, begins from target locations and works backwards to identify input conditions that can reach those targets. Compositional approaches combine both techniques by analyzing individual functions in isolation and then reasoning about their interactions.

\section{Program Vulnerability Analysis}

Software vulnerabilities represent flaws in program logic or implementation that can be exploited by malicious actors to compromise system security. Understanding these vulnerabilities is crucial for developing effective analysis techniques that can detect them before deployment.

Traditional testing approaches often fail to discover these vulnerabilities because they typically occur only under specific input conditions that are unlikely to be encountered through random testing. Static analysis can identify potential vulnerabilities but often produces high false positive rates due to conservative approximations required for soundness. Dynamic analysis provides precise information about actual program execution but is limited to the specific inputs and execution paths exercised during testing.

Symbolic execution addresses these limitations by systematically exploring multiple execution paths and generating inputs that trigger different program behaviors. However, the path explosion problem means that uniform exploration strategies may spend significant computational resources on paths that are unlikely to contain security vulnerabilities. This motivates the development of security-focused analysis techniques that prioritize exploration of paths involving user-controlled data, as these represent the primary attack vectors for most software vulnerabilities.

\section{Taint Analysis}

Taint analysis tracks the propagation of data derived from untrusted sources throughout program execution. Data originating from designated sources (such as user input functions like \texttt{fgets}, \texttt{gets}, \texttt{read}, or \texttt{scanf}) is marked as tainted. The analysis tracks how this tainted data flows through assignments, function calls, and other operations. When tainted data reaches a security-sensitive sink (such as buffer operations or system calls), the analysis flags a potential vulnerability.

The propagation rules define how taint spreads through different operations: assignments involving tainted values result in tainted variables, arithmetic operations with tainted operands typically produce tainted results, and function calls with tainted arguments may result in tainted return values depending on the function's semantics. Dynamic taint analysis performs tracking during program execution, providing precise information about actual data flows while considering specific calling contexts and program states, resulting in reduced false positives compared to static analysis approaches.

\section{Control Flow Analysis}

Control flow analysis constructs and analyzes control flow graphs (CFGs) representing program structure. CFG nodes correspond to basic blocks of sequential instructions and edges represent possible control transfers between blocks. This representation enables systematic analysis of program behavior and reachability properties.

Static analysis constructs CFGs by examining program code without execution, analyzing structure and control flow based solely on the source code or binary representation. This approach offers comprehensive coverage and efficiency, enabling examination of all statically determinable program paths without requiring specific input values. However, static analysis faces limitations including difficulty with indirect call resolution and potential false positives due to conservative approximations required for soundness.

Dynamic analysis executes the program and collects runtime information, providing precise information about actual program behavior and complete execution context. This approach eliminates many false positives inherent in static analysis and validates that control flow relationships are actually exercised under realistic conditions. However, dynamic analysis results depend heavily on input quality and coverage.

A Call Graph represents function call relationships within a program, where each node corresponds to a function and each directed edge represents a call relationship. Call graphs serve important purposes including program understanding, entry point identification, reachability analysis, and complexity assessment. Call graphs prove valuable for path prioritization strategies, enabling identification of functions reachable from tainted input sources and assessment of their relative importance in program execution flow.

\section{Angr Framework}

Angr is an open-source binary analysis platform providing comprehensive capabilities for static and dynamic program analysis~\cite{shoshitaishvili_sok_2016}. The platform supports multiple architectures and provides a Python-based interface for research and education~\cite{springer_teaching_2018}. Key components include the \textit{Project} object representing the binary under analysis with access to contents, symbols, and analysis capabilities; the \textit{Knowledge Base} storing information gathered during analysis including function definitions and control flow graphs; the \textit{Simulation Manager} handling multiple program states during symbolic execution and managing state transitions; and the \textit{Solver Engine} interfacing with constraint solvers to determine path feasibility and solve for concrete input values.

Angr supports both static (\texttt{CFGFast}) and dynamic (\texttt{CFGEmulated}) CFG construction. Static analysis provides efficiency but may miss indirect calls, while dynamic analysis offers completeness at higher computational cost. The framework represents program states with register values, memory contents, path constraints, and execution history, providing APIs for state manipulation and exploration control through step functions and various exploration strategies including depth-first search, breadth-first search, and custom heuristics.

The framework's extensible architecture enables integration of custom analysis techniques, making it particularly suitable for implementing novel symbolic execution optimizations. The symbolic execution landscape includes numerous frameworks targeting different domains and applications, ranging from language-specific tools like KLEE\footnote{https://klee-se.org/} for LLVM\footnote{https://llvm.org/} bitcode to specialized platforms for smart contract analysis. Angr's comprehensive binary analysis capabilities, multi-architecture support, and extensible Python-based architecture make it well-suited for implementing taint-guided exploration strategies.
