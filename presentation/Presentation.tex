\documentclass[aspectratio=169]{beamer}

% University of Basel Beamer Style
\usepackage{unibas-beamer}

% Presentation metadata
\title{TraceGuard: Taint-Guided Symbolic Execution}
\subtitle{Bachelor Thesis Defense}
\author{Ruben Hutter}
\date{\today}
\institute[CS]{University of Basel, Faculty of Science \\ Department of Mathematics and Computer Science}

% University logo (adjust path as needed)
\titlegraphic{%
    \vspace*{4.2cm}\hspace*{0.70\linewidth}%
    \includegraphics[width=4cm]{./thm/picLogo}
}

\begin{document}

\maketitle

\section{Introduction}

\setbeamercolor{background canvas}{bg=UnibasWhitePicture}
\begin{frame}
    \frametitle{The Path Explosion Problem}
    \begin{columns}
        \begin{column}{0.6\textwidth}
            \begin{itemize}
                \item Symbolic execution explores all possible program paths
                \item Number of paths grows exponentially with program complexity
                \item Traditional approaches treat all paths equally
                \item Security vulnerabilities often occur in specific paths
                \item Need for intelligent path prioritization
            \end{itemize}
        \end{column}
        \begin{column}{0.4\textwidth}
            \begin{figure}[H]
                \centering
                % Add your path explosion diagram here
                % \includegraphics[width=0.9\linewidth]{./figures/path_explosion}
                \textit{[Path Explosion Diagram]}
            \end{figure}
        \end{column}
    \end{columns}
\end{frame}

\begin{frame}
    \frametitle{Research Motivation}
    \begin{research}{Problem Statement}
        \begin{itemize}
            \item Symbolic execution suffers from exponential path explosion
            \item Uniform exploration wastes resources on irrelevant paths
            \item Security analysis requires focus on user-input processing paths
        \end{itemize}
    \end{research}
    
    \begin{implementation}{Solution Approach}
        \begin{itemize}
            \item Integrate taint analysis with symbolic execution
            \item Prioritize paths that process potentially malicious data
            \item Guide exploration toward security-relevant code regions
        \end{itemize}
    \end{implementation}
\end{frame}

\section{Background}

\begin{frame}
    \frametitle{Symbolic Execution}
    \begin{itemize}
        \item Program analysis technique using symbolic variables
        \item Explores multiple execution paths simultaneously
        \item Generates constraints for path conditions
        \item Powerful for vulnerability discovery but suffers from scalability issues
    \end{itemize}
    
    \vspace{1em}
    \textbf{Key Challenge:} Path explosion makes exhaustive analysis intractable
\end{frame}

\begin{frame}
    \frametitle{Taint Analysis}
    \begin{itemize}
        \item Tracks data flow from untrusted sources (taint sources)
        \item Monitors propagation through program operations
        \item Identifies when tainted data reaches critical operations (sinks)
        \item Provides security-focused view of program behavior
    \end{itemize}
    
    \vspace{1em}
    \textbf{Key Insight:} Security vulnerabilities are more likely in paths processing tainted data
\end{frame}

\section{TraceGuard Approach}

\begin{frame}
    \frametitle{Core Methodology}
    \begin{enumerate}
        \item \textbf{Taint Source Recognition:} Hook input functions (fgets, scanf, read)
        \item \textbf{Dynamic Taint Tracking:} Monitor data flow through symbolic execution
        \item \textbf{Path Prioritization:} Score states based on taint interaction
        \item \textbf{Guided Exploration:} Focus resources on high-priority paths
    \end{enumerate}
    
    \vspace{1em}
    \begin{evaluation}{Result}
        Transform uniform exploration into security-focused analysis
    \end{evaluation}
\end{frame}

\begin{frame}
    \frametitle{Implementation Architecture}
    \begin{columns}
        \begin{column}{0.5\textwidth}
            \textbf{Core Components:}
            \begin{itemize}
                \item TraceGuard Class
                \item Function Hooking System
                \item Taint Tracking Engine
                \item Custom Exploration Technique
                \item Visualization Integration
            \end{itemize}
        \end{column}
        \begin{column}{0.5\textwidth}
            \textbf{Built on Angr Framework:}
            \begin{itemize}
                \item Binary analysis platform
                \item Symbolic execution engine
                \item Multi-architecture support
                \item Extensible Python interface
            \end{itemize}
        \end{column}
    \end{columns}
\end{frame}

\section{Evaluation}

\begin{frame}
    \frametitle{Experimental Setup}
    \begin{implementation}{Benchmark Suite}
        \begin{itemize}
            \item 7 synthetic test programs with known vulnerabilities
            \item Programs designed to challenge symbolic execution
            \item Multiple runs with 120-second timeout per execution
            \item Comparison: TraceGuard vs Classical Angr
        \end{itemize}
    \end{implementation}
    
    \begin{evaluation}{Metrics}
        \begin{itemize}
            \item Vulnerability detection rate
            \item Execution time performance
            \item Basic block coverage efficiency
            \item State exploration patterns
        \end{itemize}
    \end{evaluation}
\end{frame}

\begin{frame}
    \frametitle{Key Results}
    \begin{columns}
        \begin{column}{0.5\textwidth}
            \textbf{Vulnerability Detection:}
            \begin{itemize}
                \item 100\% detection rate across all tests
                \item 5× improvement in challenging scenarios
                \item Consistent performance across multiple runs
            \end{itemize}
            
            \vspace{1em}
            \textbf{Efficiency:}
            \begin{itemize}
                \item Competitive execution times
                \item 36.8\% to 75.0\% of classical coverage
                \item Focused exploration strategy
            \end{itemize}
        \end{column}
        \begin{column}{0.5\textwidth}
            \begin{figure}[H]
                \centering
                % Add your results chart here
                % \includegraphics[width=\linewidth]{./figures/results_chart}
                \textit{[Results Visualization]}
            \end{figure}
        \end{column}
    \end{columns}
\end{frame}

\section{Contributions}

\begin{frame}
    \frametitle{Research Contributions}
    \begin{research}{Novel Integration}
        \begin{itemize}
            \item First comprehensive framework for real-time taint-guided symbolic execution
            \item Dynamic state prioritization based on security relevance
            \item Practical implementation demonstrating feasibility
        \end{itemize}
    \end{research}
    
    \begin{implementation}{Technical Achievements}
        \begin{itemize}
            \item Custom Angr exploration technique
            \item Function-level taint tracking system
            \item Adaptive scoring algorithm with configurable thresholds
            \item Comprehensive benchmarking infrastructure
        \end{itemize}
    \end{implementation}
\end{frame}

\section{Future Work}

\begin{frame}
    \frametitle{Limitations and Future Directions}
    \textbf{Current Limitations:}
    \begin{itemize}
        \item Evaluation limited to synthetic test programs
        \item Primary focus on AMD64 C/C++ binaries
        \item Dependency on accurate taint source identification
    \end{itemize}
    
    \vspace{1em}
    \textbf{Future Enhancements:}
    \begin{itemize}
        \item Real-world application validation
        \item Multi-architecture and language support
        \item Enhanced taint granularity (byte-level tracking)
        \item Integration with fuzzing frameworks
        \item Machine learning-guided exploration strategies
    \end{itemize}
\end{frame}

\section{Conclusion}

\setbeamercolor{background canvas}{bg=UnibasWhitePicture}
\begin{frame}
    \frametitle{Conclusion}
    \begin{evaluation}{Achievements}
        \begin{itemize}
            \item Successfully integrated taint analysis with symbolic execution
            \item Demonstrated significant improvements in vulnerability discovery
            \item Maintained competitive performance while reducing exploration scope
            \item Provided foundation for security-aware program analysis
        \end{itemize}
    \end{evaluation}
    
    \vspace{1em}
    \textbf{Impact:} TraceGuard transforms symbolic execution from uniform exploration into intelligent, security-focused analysis, addressing fundamental scalability challenges while improving vulnerability detection effectiveness.
\end{frame}

\section{Questions?}

\begin{frame}
    \frametitle{Thank You}
    \begin{center}
        \Large Questions and Discussion
        
        \vspace{2em}
        \normalsize
        TraceGuard: Taint-Guided Symbolic Execution \\
        for Enhanced Binary Analysis
        
        \vspace{1em}
        Ruben Hutter \\
        University of Basel
    \end{center}
\end{frame}

\end{document}
